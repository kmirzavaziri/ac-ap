\documentclass[a4paper, 12pt]{article}
\usepackage{fullpage}

\usepackage{amsmath}
\usepackage{amssymb}
\usepackage{amsfonts}
\usepackage{amsthm}
\usepackage{mathtools}
\usepackage{soul}

\usepackage{color}
\usepackage{listings}

\definecolor{mygreen}{rgb}{0.0, 0.5, 0.0}
\definecolor{emphamber}{rgb}{1.0, 0.75, 0.0}
\definecolor{amber}{rgb}{1.0, 0.49, 0.0}
\definecolor{ballblue}{rgb}{0.13, 0.67, 0.8}

\lstdefinestyle{kamyarStyle}{
    basicstyle=\footnotesize\ttfamily,
    keywordstyle=\color{amber},
    stringstyle=\color{mygreen},
    commentstyle=\color{gray},
    breakatwhitespace=false,
    breaklines=true,
    keepspaces=true,
    showspaces=false,
    showstringspaces=false,
    showtabs=false,
}

\lstset{style=kamyarStyle}
\lstset{escapeinside={(*@}{@*)}}

\usepackage{xepersian}

\settextfont{XB Zar}

\theoremstyle{definition}
\newtheorem{solution}{پاسخ}[section]
\newtheorem{solutionsection}{مورد}[solution]
\renewcommand{\thesolution}{\arabic{solution}}

\renewcommand{\baselinestretch}{1.3}

\definecolor{OliveGreen}{rgb}{0.84,1,0.64}

\begin{document}

\textbf{حسابگری زیستی؛}

\textbf{حل مسئله تخصیص به کمک بهینه‌سازی کلونی مورچه؛}

\textbf{کامیار میرزاوزیری؛ 610396152}

\hrulefill

\section{خواندن مسئله}

پیش از هر کاری لازم است مسئله را از فایل ورودی بخوانیم و آن را پارس کنیم تا بتوانیم هزینهٔ لازم برای انجام تسک
$i$-ام
توسط شخص
$j$-ام
را برای هر
$i$
و
$j$
داشته باشیم. برای این منظور به سادگی کد زیر را می‌نویسیم.

\LTR
\begin{lstlisting}[language=Python]
class AP:
def __init__(self, path: str):
    with open(path) as f:
        self.costs = list(map(
            lambda line: list(map(int, line.split())),
            f.readlines()
        ))
\end{lstlisting}
\RTL

برای مثال با اجرای کد زیر

\LTR
\begin{lstlisting}[language=Python]
FILE_PATH = 'job1.assign'
ap = AP(FILE_PATH)
print(ap.costs[2][1])
\end{lstlisting}
\RTL

مقدار
$10$
را خروجی می‌گیریم که نمایانگر هزینهٔ انجام کار شمارهٔ
$2$
برای شخص شمارهٔ
$1$
می‌باشد.

\section{مورچه‌ها و پاسخ مسئله}

می‌دانیم پاسخ مسئله به صورت یک لیست از کارها می‌باشد که عضو شمارهٔ
$i$-ام
آن، نمایانگر شخصی است که کار
$i$-ام
را انجام می‌دهد به طوری که هر شخص حداکثر کار را انجام می‌دهد. پس در واقع جایگشتی از اعداد صفر تا
$n - 1$
می‌باشد. لذا انتظار داریم که یک مورچه حافظه‌ای شامل تخصیص‌هایی که تا کنون انجام داده داشته باشد و در هر مرحله، کار جدیدی که به شخصی تخصیص نیافته، به یکی از اشخاص جدید تخصیص دهد.

\LTR
\begin{lstlisting}[language=Python]
class Ant:
    def __init__(self, ap: AP):
        self.ap = ap
        self.assignments = [-1] * self.ap.count
        self.cost = None
\end{lstlisting}
\RTL

\section{فورمون}

می‌توانیم تخصیص‌ها را به صورت یال‌هایی بین اشخاص و کارها در نظر بگیریم، برای هر یک از این یال‌ها مقداری فورمون در نظر می‌گیریم و این مقادیر را در کلاس
\texttt{AP}
ذخیره می‌کنیم تا برای تمامی مورچه‌ها قابل دسترسی باشد.

\LTR
\begin{lstlisting}[language=Python]
self.count = len(self.costs)
for _ in range(self.count):
    self.pheromones = [[1 / self.costs[i][j] for j in range(self.count)] for i in range(self.count)]
\end{lstlisting}
\RTL

در ابتدا مقدار فومون روی هر یال را متناسب با هزینهٔ آن یال در نظر می‌گیریم به طوری که هرچه هزینهٔ کمتری داشته باشد فورمون بیشتری دارد.

\section{تخصیص}

یک مورچه در مرحلهٔ
$k$-ام
از یافتن پاسخ باید تصمیم بگیرد که کار
$k$-ام
را به کدام شخص تخصیص دهد. که این فرایند به کمک فرمول موجود انجام می‌شود. پس به سادگی کد زیر را می‌نویسیم.

\LTR
\begin{lstlisting}[language=Python]
def assign(self, k):
    weights = [
        (self.ap.pheromones[k][i] ** ALPHA) *
        ((1 / self.ap.costs[k][i]) ** BETA)
        if i not in self.assignments[:k] else 0
        for i in range(self.ap.count)
    ]
    self.assignments[k] = random.choices(
        range(self.ap.count),
        weights=weights,
        k=1
    )[0]
\end{lstlisting}
\RTL

\section{کلونی مورچه}

یک کلونی شامل تعدادی مورچه که قابل تنظیم است به کلاس
\texttt{AP}
اضافه می‌کنیم.

\LTR
\begin{lstlisting}[language=Python]
self.colony = [Ant(self) for _ in range(COLONY_POPULATION)]
\end{lstlisting}
\RTL

\section{قدم‌ها}

برنامه را طوری طراحی می‌کنیم که در هر قدم از اجرای برنامه، تمامی مورچه‌ها هر کدام یک تخصیص را به صورت همزمان انجام دهند. برای این منظور متد
\texttt{step}
را به کلاس
\texttt{AP}
اضافه می‌کنیم.

\LTR
\begin{lstlisting}[language=Python]
def step(self, k):
    for ant in self.colony:
        ant.assign(k)
\end{lstlisting}
\RTL

\section{تبخیر}

متدی به کلاس
\texttt{AP}
اضافه می‌کنیم که وظیفهٔ تبخیر فورمون موجود بر روی یال‌ها را بر عهده بگیرد. این کار به سادگی قابل انجام می‌باشد.

\LTR
\begin{lstlisting}[language=Python]
def evaporate(self):
    for i in range(self.count):
        for j in range(self.count):
            self.pheromones[i][j] *= (1 - EVAPORATION_PARAMETER)
\end{lstlisting}
\RTL

\section{ترشح فورمون}

پس از این که یک دور کامل از قدم‌ها انجام شد و هر مورچه به یک پاسخ رسید، می‌توانیم به کمک فرمول موجود، مقداری فورمون روی هر یک از یال‌های موجود در مسیر هر مورچه ترشح کنیم. این کار را به سادگی به کمک متد زیر انجام می‌دهیم.

\LTR
\begin{lstlisting}[language=Python]
def secret_pheromones(self):
    for ant in self.colony:
        ant.calculate_cost()
        delta = 1 / ant.cost
        for i in range(self.count):
            j = ant.assignments[i]
            self.pheromones[i][j] += delta

        if self.total_best_ant_cost is not None:
            e_delta = ELITISM / self.total_best_ant_cost
            for i in range(self.count):
                j = self.total_best_ant_assignments[i]
                self.pheromones[i][j] += e_delta
\end{lstlisting}
\RTL

توجه کنید که برای محاسبهٔ هزینه، متد
\texttt{calculate\_cost}
را به کلاس
\texttt{Ant}
اضافه کردیم.

\LTR
\begin{lstlisting}[language=Python]
def calculate_cost(self):
    self.cost = sum([self.ap.costs[i][self.assignments[i]] for i in range(self.ap.count)])
\end{lstlisting}
\RTL

همچنین در این متد ترشح فورمون از روش مورچه نخبه بهره می‌بریم تا نتیجهٔ بهتری حاصل شود.

\section{دورها}

برای یافتن جواب قابل قبول لازم است که فرایند ذکرشده را بارها تکرار کنیم، برای این منظور متد
\texttt{iterate}
را در کلاس
\texttt{AP}
پیاده‌سازی می‌کنیم که هر بار اجرای آن موجب یافتن
$n$
پاسخ برای مسئله توسط هر یک از
$n$
مورچه می‌شود.

\LTR
\begin{lstlisting}[language=Python]
def iterate(self):
    for k in range(self.count):
        self.step(k)
    self.evaporate()
    self.secret_pheromones()
\end{lstlisting}
\RTL

دقت کنیم که تصمیم‌گیری مورچه‌ها در قسمت اول، اثرات جانبی ندارد و می‌توانیم آن را به صورت موازی پیاده‌سازی کنیم تا از منابع سیستم حداکثر استفاده را برده و عملکرد بهتری داشته باشیم.

\LTR
\begin{lstlisting}[language=Python]
def iterate(self):
    if PARALLEL:
        def chunk_assgin(indices, ants, rd):
            for i in range(len(indices)):
                for k in range(self.count):
                    ants[i].assign(k)
                rd[indices[i]] = ants[i].assignments

        chunks = numpy.array_split(range(len(self.colony)), P_COUNT)
        manager = mp.Manager()
        rd = manager.dict()
        processes = [
            mp.Process(target=chunk_assgin, args=(chunk, self.colony[chunk[0]:chunk[-1] + 1], rd))
            for chunk in chunks
        ]

        for p in processes:
            p.start()
        for p in processes:
            p.join()

        for i in range(len(self.colony)):
            self.colony[i].assignments = rd[i]
    else:
        for k in range(self.count):
            self.step(k)

    self.evaporate()
    self.secret_pheromones()
\end{lstlisting}
\RTL

برای این منظور پارامتر
\texttt{P\_COUNT}
را به اندازه تعداد هسته‌های
\texttt{CPU}
سیستمی که برنامه روی آن اجرا می‌شود در نظر می‌گیریم، و کلونی را به این تعداد گروه تقسیم می‌کنیم و وظیفهٔ محاسبهٔ تخصیص‌های هر گروه از مورچه‌ها را به یک هسته محول می‌کنیم.

\section{حل مسئله}

در نهایت لازم است که چند بار کل این فرایند را تکرار کنیم تا پاسخ خوبی به دست بیاوریم. برای این منظور متد
\texttt{solve}
را به کلاس
\texttt{AP}
اضافه می‌کنیم.

\LTR
\begin{lstlisting}[language=Python]
def solve(self):
    i = 0
    stagnancy = 0
    while True:
        self.iterate()
        i += 1

        new_best_ant = min(self.colony, key=lambda ant: ant.cost)
        if self.total_best_ant_cost is not None:
            if self.total_best_ant_cost - new_best_ant.cost < IMPROVEMENT_THRESHOLD:
                stagnancy += 1
                if stagnancy >= STAGNANCY_THRESHOLD:
                    break
            else:
                stagnancy = 0

        if self.total_best_ant_cost is None or new_best_ant.cost < self.total_best_ant_cost:
            self.total_best_ant_cost = new_best_ant.cost
            self.total_best_ant_assignments = new_best_ant.assignments.copy()

    return [self.total_best_ant_cost, self.total_best_ant_assignments]
\end{lstlisting}
\RTL

همانطور که طبق کد واضح است برای شرط توقف از دو مفهوم بهبود و رکود کمک گرفتیم، به این صورت که اگر بهترین پاسخ بعد از تعداد مشخصی دور، هر بار بهبود اندکی داشته باشد، الگوریتم را پایان می‌دهیم.

نهایتا کد زیر را برای حل مسئله اجرا می‌کنیم.

\LTR
\begin{lstlisting}[language=Python]
ap = AP(FILE_PATH)

t_start = time.time()
answer = ap.solve()
t_end = time.time()

print(f'Took: {t_end - t_start:.5f} seconds')
print(f'Cost: {answer[0]}')
print('Assignments:', end=' ')
print(answer[1])
\end{lstlisting}
\RTL

\section{گزارشات}
حال کد بالا را برای هر یک از آزمایش‌های موجود در تمرین اجرا می‌کنیم و در هر مورد با آزمون‌وخطا پارامترها را طوری پیدا می‌کنیم که الگوریتم سریع‌تر به پاسخ برسد.

\subsection{آزمایش اول}
پارامترها را به صورت زیر مقداردهی می‌کنیم و کد را اجرا می‌کنیم.

\LTR
\lstinputlisting[language=Python]{job1.py}
\RTL

خروجی زیر را دریافت می‌کنیم.

\LTR
\lstinputlisting{job1.anwser.txt}
\RTL

\subsection{آزمایش دوم}
پارامترها را به صورت زیر مقداردهی می‌کنیم و کد را اجرا می‌کنیم.


\LTR
\lstinputlisting[language=Python]{job2.py}
\RTL

خروجی زیر را دریافت می‌کنیم.

\LTR
\lstinputlisting{job2.anwser.txt}
\RTL

\subsection{آزمایش سوم}
پارامترها را به صورت زیر مقداردهی می‌کنیم و کد را اجرا می‌کنیم.


\LTR
\lstinputlisting[language=Python]{job3.py}
\RTL

خروجی زیر را دریافت می‌کنیم.

\LTR
\lstinputlisting{job3.anwser.txt}
\RTL

\end{document}